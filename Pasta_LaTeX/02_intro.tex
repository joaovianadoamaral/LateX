Introdução

Funcionamento
Um número significativo de equipamentos eletrônicos necessitam a utilização de uma fonte de tensão contínua, geralmente entre 10V e 20V. Todavia, a tensão fornecida pela rede é trifásica senoidal, de frequência 60Hz e valores de amplitude de 127Vrms (monofásico, ou tensão fase-neutro) ou 220Vrms (bifásico, ou tensão fase-fase). Logo, torna-se necessário converter essa situação para outra de tensão contínua, com o mínimo de oscilações possíveis. 
O circuito responsável pela tarefa proposta é composto por 5 partes principais.

Circuito de Retificação
Composto por um transformador monofásico com tap central e um circuito retificador de onda total capaz de gerar uma tensão Vcc admitindo pequena tensão de Ripple.

Circuito da Tensão de Referência
Circuito responsável por gerar um valor de tensão menor que Vcc, que será posteriormente utilizado para comparação no circuito de proteção e entrada do circuito de amplificação de potência.

Circuito da Fonte Auxiliar
Tem como função estabelecer uma tensão contínua também menor que Vcc, com valores positivos e negativos capazes de alimentar os Amplificadores Operacionais utilizados no projeto.

Circuito de Amplificação de Projeto
É a parte principal. Este circuito é composto por um transistor de potência na configuração Darlington, responsável por promover uma amplificação de potência em sua saída, que é tomada como a tensão de saída da fonte, como um todo.

Circuito de Proteção
Por fim, o circuito de proteção tem como função proteger a fonte contra curto-circuitos. É composto amplificadores de diferença e um elemento lógico flip flop, que promove uma alteração no circuito de referência ao se detectar um curto-circuito.