\subsection{Estilo de fontes para  destacar texto}

O destaque de um texto é consistentemente obtido usando   \emph{ênfase} (\verb|\emph{ênfase}|), que automaticamente seleciona o estilo itálico ou romano (não-itálico) de acordo com o contexto. Quando deseja-se destacar  no texto usando sempre um mesmo estilo usa-se  o   \textbf{negrito} (\verb|\textbf{negrito}|),  \textit{itálico} (\verb|\textit{itálico}|) ou \underline{sublinhado} (\verb|\underline{sublinhado}|). Código fonte ou nomes de comandos e macros são normalmente formatados com o estilo \texttt{teletipo} (\verb|\texttt{teletipo}|).

No ambiente de formato matemático o \textbf{negrito}, $\mathbf{R}$, é obtido com a macro (\verb|\mathbf{R}|), ou \emph{blackboard bold}, $\mathbb{R}$ (\verb|\mathbb{R}|) usados normalmente para descrever conjuntos de números reais $\mathbb{R}$, inteiros $\mathbb{Z}$, complexos $\mathbb{C}$, etc. 

Quando se está em modo matemático e é preciso escrever texto com fonte romana usamos a macro \verb|\text|.  Por exemplo, ao descrever uma função linear por partes denominada \texttt{rampa} usamos o ambiente \verb|case|:

\begin{multicols}{2}
\begin{equation*}
	r(x)= \begin{cases}
		2\,x & \text{se }  x \ge 0 \\
		0 & \text{se }  x < 0
	\end{cases}
\end{equation*}
	\vfill\null 
	\columnbreak
	\begin{lstlisting}[language={[Latex]Tex},frame=single]
\begin{equation*}
	r(x)= \begin{cases}
		2 \, x & \text{se }  x \ge 0 \\
		0       & \text{se }  x < 0
	\end{cases}
\end{equation*}
	\end{lstlisting}
\end{multicols}

\section{Espaços e mudança de linha}

O \LaTeX\ ignora espaços extras e quebra de linha. Por exemplo, 
\begin{lstlisting}[language={[Latex]Tex},frame=single]
Uma sentença    longa       cheia         de espaços e com quebra 
de linha é formatada     sem     os espaços    extras.
\end{lstlisting}

\fbox{Uma sentença    longa       cheia         de espaços e com quebra 
	de linha é formatada     sem     os espaços    extras.}


Salta-se uma linha vazia completa para quebrar um parágrafo em  dois. Coloca-se \verb|\\| no final de uma linha para criar uma nova linha, mas sem iniciar um novo parágrafo.

Usa-se  \verb|\noindent| para impedir a indentação de um novo parágrafo.

\section{Comentários}

Usa-se o caracter ``\%''   para comentar uma linha. Nada é formatado à direita  do caracter ``\%'' numa linha.

 \verb|$f(t)=\sin(\omega\, t)$. \% esta é uma função senoidal| 
 
 resulta em $f(t)=\sin(\omega\, t)$. % esta é a função senoidal

\section{Caracteres delimitadores}

\begin{tabular}{lllll}
	\toprule
	\emph{Descrição} & \emph{Comando}  & \emph{Resultado} &\emph{Objetos grandes} & \emph{Resultado}\\
	\midrule
	Parêntesis &\verb|(x)| & (x) &\verb|$\qty(\dfrac{1}{\tau\, s})$| & $\qty(\dfrac{1}{\tau\, s})$\\
	Colchetes &\verb|[x]| & [x]&\verb|$\qty[\dfrac{1}{\tau\, s}]$| & $\qty[\dfrac{1}{\tau\, s}]$\\
	Chaves & \verb|\{x\}| & \{x\}&\verb|$\qty{\dfrac{1}{\tau\, s}}$$| & $\qty{\dfrac{1}{\tau\, s}}$\\
	\bottomrule
\end{tabular}

Para tornar os delimitadores grandes o suficiente para abraçar o conteúdo, use-os junto com \verb|\right| e \verb|\left|. Usando o pacote \texttt{physics},  a macro \verb|\qty()| formata corretamente objetos de qualquer tamanho, simplificando a edição.

O par de chaves $\qty{}$ são caracteres invisíveis usados para agrupar texto e objetos formados por mais de um caracter. Observe as diferenças nas seguintes expressões
 \verb|x^2|, \verb|x^{2}|, \verb|x^2t|, \verb|x^{2t}| quando formatadas: $x^2$, $x^{2}$, $x^2t$, $x^{2t}$.


\section{Listas}

Listas são editadas de duas maneiras: ordenada (\verb|enumerate|) ou não ordenada (\verb|itemize|). Em ambos os casos é importante manter o paralelismo de linguagem, i.e. se o item começar com um verbo, todos os demais devem ser verbos também. Se escolher um substantivo, mantenha todos os itens iniciando com substantivos. O nosso cérebro não aprecia quebra de paralelismo de linguagem em listas e também em enumerações no meio do texto também!

\begin{multicols}{2}
	As três etapas essenciais em um sistema de automação são:
		\begin{enumerate}
			\item Energizar
			\item Partir
			\item Parar
		\end{enumerate}
 \vfill\null 
\columnbreak
	\begin{lstlisting}[language={[Latex]Tex},frame=single]
	As três etapas essenciais em um 
	sistema de automação são:
\begin{enumerate}
	\item Energizar
	\item Partir
	\item Parar
\end{enumerate}
	\end{lstlisting}
\end{multicols}

\begin{multicols}{2}
Avaliamos para:
		\begin{itemize}
	\item Conhecer
	\item Valorizar
	\item Responsabilizar
\end{itemize}

  \columnbreak
	\begin{lstlisting}[language={[Latex]Tex},frame=single]
		Avaliamos para:
		\begin{itemize}
			\item Conhecer
			\item Valorizar
			\item Responsabilizar
		\end{itemize}
	\end{lstlisting}
\end{multicols}


