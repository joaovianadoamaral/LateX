% !TEX encoding = UTF-8 Unicode
\chapter{Conclusões e sugestões de trabalho futuro\label{cap6}}

\section{Conclusões}
 Os objetivos foram alcançados? Observe as recomendações a seguir na seção~ \ref{sec:dicasConclusao}:  \emph{dicas sobre redação de conclusão}.
 
 
\section{Propostas de Trabalho futuro.}

É oportuno sugerir tópicos para a continuidade do trabalho, demonstrando compreensão de aspectos relevantes que transcendem o escopo apresentado. Isso demonstra que você não fez tudo que era possível e que antevê coisas interessantes para serem feitas futuramente. Contudo, evite exageros para não transparecer que o projeto como proposto e o que realmente importa ainda está por ser feito!


\section{Dicas sobre redação de Conclusão}\label{sec:dicasConclusao}

\paragraph{O que escrever na seção Conclusões?}
A palavra conclusão tem dois significados: 
\begin{itemize}
	\item refere-se às inferências extraídas de dados técnicos; 
	\item refere-se a última parte ou seção de um documento, também conhecido como comentários finais.
\end{itemize}

Conclusões são inferências extraídas de fatos ou dados que explicam os resultados. Recomenda-se apresentar análise de resultados e conclusões separadamente. Além de análise de resultados e conclusões em geral há ainda espaço para algumas recomendações e sugestões futuras, apresentadas em seção separadamente também. A separação entre seções facilita o entendimento de um raciocínio ou lógica analítica (i.e. raciocínio por partes). 
Conclusões e recomendações são sequências óbvias e lógicas a partir de resultados, entretanto, é sempre desejável apresentar algumas explicações.

Ao concluir um documento é importante auxiliar o leitor respondendo as perguntas:
\begin{enumerate}
	\item Quais são os principais tópicos e objetivos apresentados no documento? Tudo que foi proposto como objetivo deve ser discutido na conclusão. Se algum objetivo não foi alcançado, provavelmente deveria ter sido declinado da lista de objetivos ou houve alguma dificuldade. Você pode sugerir modificações nos objetivos desde que devidamente justificadas.
	\item O que fazer após ler o relatório? Mesmo que o relatório tenha sido concluído, é interessante oferecer recomendações de continuidade ou extensão em trabalho futuro. Concluir com uma oferta de novas ideias e sugestões de trabalhos futuros é uma boa ideia para demonstrar entendimento. Não exagere, pois incorre-se no risco de passar a impressão que o que foi feito é irrelevante!
\end{enumerate}

\paragraph{Sugestões para autoavaliação das conclusões:}
\begin{enumerate}
	\item Suas conclusões destacam 
	\begin{enumerate}
			\item os principais pontos discutidos, 
			\item os objetivos alcançados, 
			\item recomendações de algo a seguir, 
			\item informações adicionais, 
			\item sugestão de continuidade e aprimoramento da atividade no futuro.
	\end{enumerate}
	\item Suas conclusões decorrem claramente dos resultados obtidos e analisados?
	\item Suas recomendações ou sugestões de trabalho futuro são relevantes e decorrem claramente das conclusões?
\end{enumerate}

As partes mais importantes do relatório são suas análises, conclusões e recomendações, i.e. o que se descobriu e o que se recomenda ser feito a respeito. Se estas seções estiverem mal redigidas (uma joça!), seguramente a avaliação do relatório será prejudicada.
