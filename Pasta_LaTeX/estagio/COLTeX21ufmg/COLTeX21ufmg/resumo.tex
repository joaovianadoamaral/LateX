% !TEX encoding = UTF-8 Unicode
%\section{Abstract}

O resumo deve explicitar o assunto, a proposta e o escopo do trabalho. Ao ler o resumo o leitor deve entender de que se trata efetivamente o trabalho ou projeto, o que se aborda e como se produz evidências (simulação, experimento, dedução teórica). Normalmente a última sentença do resumo trata das evidências e discussões apresentadas ou almejadas. 

Neste compêndio de recomendações são apresentadas ideias e roteiros para se elaborar desde uma proposta de um projeto de trabalho técnico e.g. de final de curso ou estágio curricular, até o texto final do relatório ou monografia. Após apresentar uma estrutura típica de textos de relatos técnicos, são apresentadas dicas de uso do formatador de texto \LaTeX\ para facilitar e agilizar a edição de equações, figuras, tabelas e respectivas referências cruzadas. Em seguida, apresentam-se recomendações para elaboração e formatação de uma proposta de trabalho técnico com dicas para formatar um plano de trabalho e relacionar os recursos inicialmente necessários. Uma metodologia com dicas para se proceder durante a execução de um projeto são fornecidas. Para ajudar a manter o foco do desenvolvimento do trabalho e da redação técnica do mesmo, apresenta-se uma relação de critérios típicos de avaliação de trabalhos técnicos tanto da parte escrita quanto da apresentação em um seminário. Conhecer como é avaliado um trabalho técnico  ajuda a manter o foco no que é relevante para se otimizar a qualidade do trabalho relatado. Nas conclusões são fornecidas dicas  para se redigir o capítulo de conclusão do trabalho.

\arb[PARA DELETAR:]{Exemplo de comentário marcado como PARA DELETAR. Para editar o texto do resumo faça-o no arquivo em separado: \texttt{resumo.tex}  }
